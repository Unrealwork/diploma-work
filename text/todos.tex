\documentclass[a4paper]{article}
\usepackage{fontspec}
\setmainfont[Mapping=tex-text]{CMU Serif}
\usepackage{polyglossia}
\setdefaultlanguage{russian}


\usepackage{ulem}

\begin{document}

\section*{Отчёт}
\begin{itemize}
  \item \sout{исправить формулы ДПУ и ОДПУ (поменять нормирование)}
  \item \sout{дописать про то, что при сжатии хранится свой коэффициент для каждого окна}
  
  \item \sout{добавить исходников}
  \item \sout{fix STWT/iSTWT definition}
  \item \sout{выпилить newpage}
  \item \sout{полные ссылки}
  \item \sout{pagetotal в списке литературы}
  \item \sout{написать про семпл с голосом (вставить в табличку)}
  \item \sout{разделить сигнал и поток данных}
  \item \sout{ссылка на Уолша}
\end{itemize}




\section*{Презо}
\begin{itemize}
  \item \sout{добавить цитирований}
  \item \sout{добавить <<спасибо>> слайд}
  \item \sout{добавить степени и должности}
  \item \sout{сделать футер "n of N"}
  \item \sout{слайд 8: убрать лишнее}
  \item \sout{отчество}
  \item \sout{пример rev}
  \item \sout{определение фигурной скобки (побитовый XOR)}
  \item \sout{$w_{rect}$ запятые после 1 и 0}
  \item \sout{оконная функция == окно}
  \item \sout{"использовать окна будем потом" или сделать reordering}
  \item \sout{обращение быстрого уолша добавить про скорость}
  \item группировка элементов списка
  \item урезать до 15 слайдов/5 минут
\end{itemize}





\section*{Остальное}
\begin{itemize}
  \item {подготовить звуки для демонстрации}
  \item \sout{собрать jar'ники для демонстрации}
  \item взять с собой колонку на предзащиту
\end{itemize}






\end{document}