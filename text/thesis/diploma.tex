% Тут используется класс, установленный на сервере Papeeria. На случай, если
% текст понадобится редактировать где-то в другом месте, рядом лежит файл matmex-diploma-custom.cls
% который в момент своего создания был идентичен классу, установленному на сервере.
% Для того, чтобы им воспользоваться, замените matmex-diploma на matmex-diploma-custom
% Если вы работаете исключительно в Papeeria то мы настоятельно рекомендуем пользоваться
% классом matmex-diploma, поскольку он будет автоматически обновляться по мере внесения корректив
%
\documentclass{matmex-diploma-custom}

\usepackage{amsfonts}
\usepackage{amsmath}
\usepackage{hyperref}
\usepackage{listings}
\lstset{language=Java, captionpos=b, basicstyle=\footnotesize, breakatwhitespace=true, breaklines=true}
\renewcommand{\lstlistingname}{Код программы}

\begin{document}
\filltitle{ru}{
    chair              = {Кафедра Информатики},
    title              = {Адаптивная реализация алгортма сжатия JPEG изображений PackJPG с помошью технологии OpenCL},
    type               = {diploma},
    position           = {студента},
    group              = 461,
    author             = {Шмагринский Игорь Олегович},
    supervisorPosition = {к.\,ф.-м.\,н., доцент},
    supervisor         = {{\sigplace{Н.Ю.~Ловягин}{подпись}}},
    reviewerPosition   = {д.\,ф.-м.\,н., ассистент},
    reviewer           = {{\sigplace{Н.Н.~Неизвестный}{подпись}}},
    chairHeadPosition  = {д.\,ф.-м.\,н., профессор},
    chairHead          = {{\sigplace{В.О.~Костин}{подпись}}},
%   university         = {Санкт-Петербургский Государственный Университет},
%   faculty            = {Математико-механический факультет},
%   city               = {Санкт-Петербург},
%   year               = {2013}
}
\filltitle{en}{
    chair              = {Chair of Computer Science},
    title              = {Adaptive implementation of JPEG image compression algorithm PackJPG with OpenCL technology},
    author             = {Igor Shmagrinsky},
    supervisorPosition = {docent},
    supervisor         = {{\sigplace{N.Y.~Lovyagin}{signature}}},
    reviewerPosition   = {professor},
    reviewer           = {{\sigplace{N.N.~Neizvestny}{signature}}},
    chairHeadPosition  = {professor},
    chairHead          = {{\sigplace{V.O~Kostin}{signature}}},
}
\maketitle
\tableofcontents
\newpage
% У введения нет номера главы
\section*{Введение}
IT - мир не стоит на  месте и развивается с неимоверной скоростью, сейчас трудно представить, что когда-то в большинстве устройств был всего лишь один процессор. Сейчас даже в мобильных аппаратах присутствует более чем одно вычислительное устройство. Это позволяет выполнять трудоемкие операции в несколько раз быстрее,но многие в наше время забывают о распараллеливании, поэтому большинство программ работают неэффективно и использую не всю мощь современных устройств. В данной работе расмотрен один из эффективнейших алгоритмов сжатия изображения PackJPG, который уступает другим алгоритмам своей области лишь по времени работы. В ходе исследования будет проведено исследование возможности адаптации данного алгоритма и его реализация на процессорах видеокарт с помощью технологи OpenCL. Так же будет выполненма оценка времени работы алгоритма, его реальное время и сравнение трудоемкости существующих алгоритмов сжатия изображений и  нашей реализации.
\section{Предварительные сведения}
Сжатие это техника направленная на понижение размерности данных без  чрезмерной потери качества мультимедийных данных. 
Перемещение и хранение сжатых мультимединых  файлов намного быстрее  и эффективнее, чем у оригинальных данных. Если при сжатии уменьщается снижается качество исходных данных то это сжатие с потерями, иначе - без потерь. Существует множестнво разнообразных  техник и стандартов для сжатия мультемедийных данных с потерями, одним из них на протяжении многих лет является формат изображений JPEG.
\subsection{Стандарт сжатия изображений JPEG}
JPEG-это стандарт сжатия изображений разработаный Joint Photogrphic Experts Group. Он был официально одобрен мировым сообществом в 1992 году. JPEG содержит некоторое число шагов, каждый из которых способствует процессу сжатия.
Рассмотрим каждый из них детально.
\subsubsection{Кодирование}
Хотя \textit{JPEG} файл может закодирован разными путями, но самый известный и популярный - \textit{JFIF} кодировние. Процесс кодирования сстоит из нескольких шагов.\\
\paragraph{Преобразование цветового пространства} \\
Многие цвета изображения представлены с помощью цветового пространства \textit{RGB}. Такое представление, однако, сильно коррелирует, что подразумевает, что это цветовое пространство не очень подходит для независимого кодирования. Так как человеческая зрительная система менее чувствительна к позиции и движению ярких  цветов. \\
Следовательно,целесообразней изпользовать цветовое пространство \textit{YCrBr}. \\
\textbf{Разбиение исходного изображения}\\
\textbf{Дискретное косинусное преобразование}\\
\textbf{Квантование}\\
\textbf{Энтропийное кодирование}\\
\subsubsection{Декодирование}
\subsubsection{Представление}
\subsection{Сжатие JPEG изображений без потерь}
\subsection{Технология OpenCL}
\section{Алгоритм PackJPG}
\subsection{Основная идея}
\subsection{Коды Хаффмена}
\subsection{Арифмитическое кодирование}
\section{Адаптивная реализация с помощью технологии OpenCL}
\section{Результаты}
\section{Демо приложения}

% У заключения нет номера главы
\section*{Заключение}

\bibliographystyle{ugost2008}
\bibliography{diploma.bib}
\end{document}