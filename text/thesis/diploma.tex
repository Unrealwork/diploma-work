% Тут используется класс, установленный на сервере Papeeria. На случай, если
% текст понадобится редактировать где-то в другом месте, рядом лежит файл matmex-diploma-custom.cls
% который в момент своего создания был идентичен классу, установленному на сервере.
% Для того, чтобы им воспользоваться, замените matmex-diploma на matmex-diploma-custom
% Если вы работаете исключительно в Papeeria то мы настоятельно рекомендуем пользоваться
% классом matmex-diploma, поскольку он будет автоматически обновляться по мере внесения корректив
%
\documentclass{matmex-diploma-custom}

\usepackage{amsfonts}
\usepackage{amsmath}
\usepackage{hyperref}

\usepackage{listings}
\lstset{language=Java, captionpos=b, basicstyle=\footnotesize, breakatwhitespace=true, breaklines=true}
\renewcommand{\lstlistingname}{Код программы}



\begin{document}
\filltitle{ru}{
    chair              = {Кафедра Информатики},
    title              = {Адаптивная реализация алгортма сжатия JPEG изображений PackJPG с помошью технологии OpenCL},
    type               = {diploma},
    position           = {студента},
    group              = 461,
    author             = {Шмагринский Игорь Олегович},
    supervisorPosition = {к.\,ф.-м.\,н., доцент},
    supervisor         = {{\sigplace{Н.Ю.~Ловягин}{подпись}}},
    reviewerPosition   = {д.\,ф.-м.\,н., ассистент},
    reviewer           = {{\sigplace{Н.Н.~Неизвестный}{подпись}}},
    chairHeadPosition  = {д.\,ф.-м.\,н., профессор},
    chairHead          = {{\sigplace{В.О.~Костин}{подпись}}},
%   university         = {Санкт-Петербургский Государственный Университет},
%   faculty            = {Математико-механический факультет},
%   city               = {Санкт-Петербург},
%   year               = {2013}
}
\filltitle{en}{
    chair              = {Chair of Computer Science},
    title              = {Adaptive implementation of JPEG image compression algorithm PackJPG with OpenCL technology},
    author             = {Igor Shmagrinsky},
    supervisorPosition = {docent},
    supervisor         = {{\sigplace{N.Y.~Lovyagin}{signature}}},
    reviewerPosition   = {professor},
    reviewer           = {{\sigplace{N.N.~Neizvestny}{signature}}},
    chairHeadPosition  = {professor},
    chairHead          = {{\sigplace{V.O~Kostin}{signature}}},
}
\maketitle
\tableofcontents
\newpage
% У введения нет номера главы
\section*{Введение}

    IT - мир не стоит на  месте и развивается с неимоверной скоростью, сейчас трудно представить, что когда-то в большинстве устройств был всего лишь один процессор. Сейчас даже в мобильных аппаратах присутствует более чем одно вычислительное устройство. Это позволяет выполнять трудоемкие операции в несколько раз быстрее,но многие в наше время забывают о распараллеливании, поэтому большинство программ работают неэффективно и использую не всю мощь современных устройств.

 В данной работе расмотрен один из эффективнейших алгоритмов сжатия изображения PackJPG, который уступает другим алгоритмам своей области лишь по времени работы. В ходе исследования будет проведено исследование возможности адаптации данного алгоритма и его реализация на процессорах видеокарт с помощью технологи OpenCL. Так же будет выполненма оценка времени работы алгоритма, его реальное время и сравнение трудоемкости существующих алгоритмов сжатия изображений и  нашей реализации.

\section{Предварительные сведения}
Сжатие это техника направленная на понижение размерности данных без  чрезмерной потери качества мультимедийных данных.
Перемещение и хранение сжатых мультимединых  файлов намного быстрее  и эффективнее, чем у оригинальных данных. Если при сжатии уменьщается снижается качество исходных данных то это сжатие с потерями, иначе - без потерь. Существует множестнво разнообразных  техник и стандартов для сжатия мультемедийных данных с потерями, одним из таких стандартов протяжении многих лет является формат изображений JPEG.
\subsection{Стандарт сжатия изображений JPEG}
JPEG-это стандарт сжатия изображений разработаный Joint Photogrphic Experts Group. Он был официально одобрен мировым сообществом в 1992 году. JPEG содержит некоторое число шагов, каждый из которых способствует процессу сжатия.
Рассмотрим каждый из них детально.
\subsubsection{Кодирование}
Хотя \emph{JPEG} файл может закодирован разными путями, но самый известный и популярный - \emph{JFIF} кодировние.
Процесс кодирования сстоит из нескольких шагов.\\

\textbf{Преобразование цветового пространства} \newline

Многие цвета изображения представлены с помощью цветового пространства \emph{RGB}. Такое представление, однако, сильно коррелирует, что подразумевает, что это цветовое пространство не очень подходит для независимого кодирования. Так как человеческая зрительная система менее чувствительна к позиции и движению ярких  цветов. \\
Следовательно, целесообразней изпользовать цветовое пространство \emph{YCrBr}.\\

\textbf{Разбиение исходного изображения}\newline

После изображение разбивается yа блоки 8x8 пикселей, с каждым из которых ведется дальнейшая работа.\\

\textbf{Дискретное косинусное преобразование}\newline

Дальше каждая компонента \emph{(Y,Cr,Br)} каждого блока преобразуеется к  частотной форме. Для  этого используется двумерное диоскретное
косинусное преобразование второго типа. Перед вычислением этого преобразования, все значения сдвигаются из положительного интервала $[0,255]$
в интервал $[-128, 127]$ вычитеанием из каждой значения идидфидф  компоненты блока 128. Это действие является
обязательным, так такой интервал значений является одним из требований для дискретного косинусного преобразования. В резултате будет полуен блок $g_{x,y}$ с которым мы будем работать дальше.\newline

Блок $ g $ преообразуется по следующему принципу:
$$ \ G_{u,v} =
    \frac{1}{4}
    \alpha(u)
    \alpha(v)
    \sum_{x=0}^7
    \sum_{y=0}^7
    g_{x,y}
    \cos \left[\frac{(2x+1)u\pi}{16} \right]
    \cos \left[\frac{(2y+1)v\pi}{16} \right],
  $$
  где:
  % TODO: дописать описание DCT преобразования %
  \begin{itemize}
  \item{$u$ - вертикальная пространственная частота для целых чисел $\ 0 \leq u < 8$}
  \item{$v$ - горизонатальная пространственная частота для целых чисел $\ 0 \leq v < 8$}
  \item{
    $\alpha(u) =
    \begin{cases}
        \frac{1}{\sqrt{2}}, & \mbox{if }u=0 \\1, & \mbox{иначе}
    \end{cases}$
    - норма необходимая  для того чтобы преобразование было ортомнормированным.
  }
  \item{
    $\ g_{x,y}$ - это значение которое содержит в себе пиксель с координатами $\ (x,y)$
  }
    $\ G_{u,v}$ - это значение которое содержит в себе пиксель с координатами $\ (u,v)$
  }
  \end{itemize}

  После преобразования можно заметить что значение $ G_{0,0} $  превосходит все остальные, его так же называют
  \emph{коэффициентом DC}. Он определяет основной тон для блока в целом. Его так же можно назвать
  \emph{постоянной компонентой}. Оставшиеся 63 коэффициента (блока 8x8) называют \emph{AC
  коэффициентами}, где AC могут быть установлены для запасных компонент.
  % TODO: уточнить про AC коэфициенты %
  Преимущество дискретного косинусного преобразования - возможность вычислить основной оттенок блока (сигнала).\\

\textbf{Квантование}\newline

Человеческий глаз  хорошо приспособлен замечать маленькие различия в яркости на относительно больших расстояниях, но плохо отличает точную силу яркости на высоких частотах. Это позволяет значительно уменьшить количество информации о высокого частотных компонентах. Данную операцию можно  осуществить простым делением значения  каждой компоненты в частотном диапозоне на константу и последующим округлением этого значения ближайшим целым числом. Это действие, единственное во всем процессе сжатия при котором происходит потеря данных, в отличии от дискретного косинусного пребразования, которое выполняет вычисления с выской точностью. Как результат, многие компоненты оказваются равны  нулю или их значение очень близко к нулю, как следствие они занимают меньше бит в памяти.

%The human eye is good at seeing small differences in brightness over a relatively large area, but not so good at distinguishing the exact strength of a high frequency brightness variation. This allows one to greatly reduce the amount of information in the high frequency components. This is done by simply dividing each component in the frequency domain by a constant for that component, and then rounding to the nearest integer. This rounding operation is the only lossy operation in the whole process (other than chroma subsampling) if the DCT computation is performed with sufficiently high precision. As a result of this, it is typically the case that many of the higher frequency components are rounded to zero, and many of the rest become small positive or negative numbers, which take many fewer bits to represent.
Подобные процессы, когда процедура построения чего-либо ведтся с помощью набора дискретных (в нашем случае целых) велечин называется \emph{квантованием}.

Элементы \emph{матрицы квантования} управляют коэфициентом сжатия. Чем больше значение компонент, тем больше будет потеря изображения в качетсве. Типичная матрица квантования (качество ухудшается примерено на 50\%, утверждена как часть стандарта JPEG), выглядит следующим образом:
%The elements in the quantization matrix control the compression ratio, with larger values producing greater compression. A typical quantization matrix (for a quality of 50% as specified in the original JPEG Standard), is as follows:
$$ Q=
     \begin{bmatrix}
      16 & 11 & 10 & 16 & 24 & 40 & 51 & 61 \\
      12 & 12 & 14 & 19 & 26 & 58 & 60 & 55 \\
      14 & 13 & 16 & 24 & 40 & 57 & 69 & 56 \\
      14 & 17 & 22 & 29 & 51 & 87 & 80 & 62 \\
      18 & 22 & 37 & 56 & 68 & 109 & 103 & 77 \\
      24 & 35 & 55 & 64 & 81 & 104 & 113 & 92 \\
      49 & 64 & 78 & 87 & 103 & 121 & 120 & 101 \\
      72 & 92 & 95 & 98 & 112 & 100 & 103 & 99
     \end{bmatrix}.
$$

Квантование коэффициентов матрицы  G, полученной на предыдущем шаге, происходит следующим образом:
% The quantized DCT coefficients are computed with
е
$$B_{j,k} = \mathrm{round} \left( \frac{G_{j,k}}{Q_{j,k}} \right) \mbox{ для } j=0,1,2,\ldots,7; k=0,1,2,\ldots,7$$

 % B_{j,k} = \mathrm{round} \left( \frac{G_{j,k}}{Q_{j,k}} \right) \mbox{ for } j=0,1,2,\ldots,7; k=0,1,2,\ldots,7

  %where G is the unquantized DCT coefficients; Q is the quantization matrix above; and B is the quantized DCT coefficients.
  Полученная матрица B, отправляется на следующий шаг, где будет произвоидиться ее энтропийное кодирование.
  %Using this quantization matrix with the DCT coefficient matrix from above results in:

\textbf{Энтропийное кодирование}\\

\emph{Энтропийное кодирование} - это специальная форма сжатия данных без потерь. Она включает в себя переобределение порядка компонент изображения в виде \emph{"зиг-зага"} дальнейшее сжаатие методом \emph{кодирования повторов}, который группирует похожие повторяющие значения вместе,
 %Entropy coding is a special form of lossless data compression. It involves arranging the image components in a "zigzag" order employing run-length encoding (RLE) algorithm that groups similar frequencies together, inserting length coding zeros, and then using Huffman coding on what is left.
\subsubsection{Декодирование}
\subsubsection{Представление}
\subsection{Сжатие JPEG изображений без потерь}
\subsection{Технология OpenCL}
\section{Алгоритм PackJPG}
\subsection{Основная идея}
\subsection{Коды Хаффмена}
\subsection{Арифмитическое кодирование}
\section{Адаптивная реализация с помощью технологии OpenCL}
\section{Результаты}
\section{Демо приложения}

% У заключения нет номера главы
\section*{Заключение}

\bibliographystyle{ugost2008}
\bibliography{diploma.bib}
\end{document}