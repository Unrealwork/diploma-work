\begin{lstlisting}[caption={Вычисление дискретных функций Уолша}]
public class Walsh {
  private final int myS;

  public Walsh(int s) {
    this.myS = s;
  }

  public Function<Integer, Integer> v(final int k) {
    return new Function<Integer, Integer>() {
      @Override
      public Integer apply(Integer j) {
        return bitXor(k, j, myS) % 2 == 0 ? 1 : -1;
      }
    };
  }

  /**
   * {k,j}_v
   */
  public static int bitXor(int k, int j, int v) {
    BitSet kBits = convert(k);
    BitSet jBits = convert(j);

    int result = 0;
    for (int i = 0; i < v; i++) {
      int kBit = kBits.get(i) ? 1 : 0;
      int jBit = jBits.get(i) ? 1 : 0;
      result += kBit * jBit;
    }
    return result;
  }

  private static BitSet convert(int value) {
    BitSet bits = new BitSet();
    int index = 0;
    while (value != 0) {
      if (value % 2 != 0) {
        bits.set(index);
      }
      ++index;
      value = value >>> 1;
    }
    return bits;
  }

  private static int convert(BitSet bits) {
    int value = 0;
    for (int i = 0; i < bits.length(); ++i) {
      value += bits.get(i) ? (1L << i) : 0L;
    }
    return value;
  }

  /**
   * rev_s(k)
   */
  public static int rev(int s, int k) {
    BitSet kBits = convert(k);
    for (int i = 0, revLength = s / 2; i < revLength; i++) {
      boolean tmp = kBits.get(i);
      int idx = s - i - 1;
      kBits.set(i, kBits.get(idx));
      kBits.set(idx, tmp);
    }
    return convert(kBits);
  }
\end{lstlisting}