% \documentclass[notes=only]{beamer}
\documentclass{beamer}

\usepackage[utf8]{inputenc}
%% XeTeX preamble                                                                 
\usepackage{xltxtra}
\usepackage{hyperref}                                                           
\setmainfont[Mapping=tex-text]{DejaVu Serif}
\setsansfont[Mapping=tex-text]{DejaVu Sans}
\usepackage{polyglossia}
\usepackage{listings}
\usepackage{color}

\definecolor{mygreen}{rgb}{0,0.6,0}
\definecolor{mygray}{rgb}{0.5,0.5,0.5}
\definecolor{mymauve}{rgb}{0.58,0,0.82}

\lstset{ %
  backgroundcolor=\color{white},   % choose the background color; you must add \usepackage{color} or \usepackage{xcolor}; should come as last argument
  basicstyle=\footnotesize,        % the size of the fonts that are used for the code
  breakatwhitespace=false,         % sets if automatic breaks should only happen at whitespace
  breaklines=true,                 % sets automatic line breaking
  captionpos=b,                    % sets the caption-position to bottom
  commentstyle=\color{mygreen},    % comment style
  deletekeywords={...},            % if you want to delete keywords from the given language
  escapeinside={\%*}{*)},          % if you want to add LaTeX within your code
  extendedchars=true,              % lets you use non-ASCII characters; for 8-bits encodings only, does not work with UTF-8
  frame=single,	                   % adds a frame around the code
  keepspaces=true,                 % keeps spaces in text, useful for keeping indentation of code (possibly needs columns=flexible)
  keywordstyle=\color{blue},       % keyword style
  language=Octave,                 % the language of the code
  morekeywords={*,...},            % if you want to add more keywords to the set
  numbers=left,                    % where to put the line-numbers; possible values are (none, left, right)
  numbersep=5pt,                   % how far the line-numbers are from the code
  numberstyle=\tiny\color{mygray}, % the style that is used for the line-numbers
  rulecolor=\color{black},         % if not set, the frame-color may be changed on line-breaks within not-black text (e.g. comments (green here))
  showspaces=false,                % show spaces everywhere adding particular underscores; it overrides 'showstringspaces'
  showstringspaces=false,          % underline spaces within strings only
  showtabs=false,                  % show tabs within strings adding particular underscores
  stepnumber=2,                    % the step between two line-numbers. If it's 1, each line will be numbered
  stringstyle=\color{mymauve},     % string literal style
  tabsize=2,	                   % sets default tabsize to 2 spaces
  title=\lstname                   % show the filename of files included with \lstinputlisting; also try caption instead of title
}

\setdefaultlanguage{russian} 
\newfontfamily{\cyrillicfonttt}{CMU Serif Roman}   
          
%% Commong preamble                                  
\usepackage{hyperref}                         
\usepackage{graphicx}

\usetheme{default} 
\setbeamertemplate{navigation symbols}{} 
\useoutertheme{infolines} 
\setbeamertemplate{footline}{
  \scriptsize{\hfill\insertframenumber/\inserttotalframenumber\hspace*{.2cm}}
} 

\title{Анализ возможности и эффективности параллельной реализации алгоритма packJPG}
\author[Шмагринский Игорь]{Шмагринский Игорь Олегович, 461 группа
  \\{\footnotesize Научный руководитель: к.\,ф.-м.\,н., доц. Н.Ю.~Ловягин}
  \\{\footnotesize Рецензент: к.\,ф.-м.\,н., доц. Т.О.~Евдокимова}
}
\institute[СПбГУ]{Санкт-Петербургский Государственный Университет\\Математико-Механический факультет\\ Кафедра Информатики}
\date{06.07.2017}
\titlegraphic{\includegraphics[width=1.2cm]{spbu-logo}}

\begin{document}
    \frame{
    \titlepage
}

\defverbatim[colored]\makeset{
\begin{lstlisting}[language=C++,basicstyle=\ttfamily,keywordstyle=\color{blue}]
void make_set(int X) {
  parent[X] = X;
}
\end{lstlisting}
}
%%%%%%%%%%%%%%%%%%%%%%%%%%%%%%%%%%%%%%%%%%%%%%%%%%%%%%%%%%%%%%%%%%%%%%%%%%%%%%%%

\begin{frame}[c]
\frametitle{Программа packJPG}

   \textbf{\emph{PackJPG}} — это пограмма, специально разработанная для дальнейшего сжатия
   JPEG изображений без потерь.

    \vspace{\baselineskip}

    \textbf{Особенности}
    \begin{itemize}
        \item{Кроссплатформенность}
        \item{Высокая степень сжатия(~ 26\%) }
        \item{Низкая скорость сжатия (~ 600 kb/сек)}
    \end{itemize}
\end{frame}

%%%%%%%%%%%%%%%%%%%%%%%%%%%%%%%%%%%%%%%%%%%%%%%%%%%%%%%%%%%%%%%%%%%%%%%%%%%%%%%%

\begin{frame}\frametitle{Постановка задачи и ее актуальность}
   \textbf{Цели}
   \begin{itemize}
     \item{Анализ существующего алгоритма packJPG и его реализации}
     \item{Параллельная модфикация исходного кода}
     \item{Анализ эффективности проведенной модификации}
   \end{itemize}
   \vspace{\baselineskip}
   \textbf{Актуальность работы}
    \begin{itemize}
        \item{Многпроцессорная архитектура современных устройств}
        \item{Высокий уровень сжатия алгоритма packJPG}
        \item{Низкая скорость сжатия алгормтма packJPG}
      \end{itemize}
\end{frame}

%%%%%%%%%%%%%%%%%%%%%%%%%%%%%%%%%%%%%%%%%%%%%%%%%%%%%%%%%%%%%%%%%%%%%%%%%%%%%%%%

\begin{frame}\frametitle{Процедура анализа}
    \textbf{Основные шаги:}
    \begin{itemize}
        \item{Анализ существующего кода проекта}
        \item{Рефакторинг существующего кода}
        \item{Поиск узких мест алгоритма}
        \item{Параллельная модификация}
        \item{Оценка эффективности осуществленной модификации}
    \end{itemize}
\end{frame}

%%%%%%%%%%%%%%%%%%%%%%%%%%%%%%%%%%%%%%%%%%%%%%%%%%%%%%%%%%%%%%%%%%%%%%%%%%%%%%%%

\begin{frame}\frametitle{Используемые технологии}
    \begin{itemize}
        \item{\textbf{\emph{OpenMP}} - API для параллельного программирования}
        \item{\textbf{\emph{Valgrind}} - набор инструментов для профайлинга и отладки программ написанных на С++}
        \item{\textbf{\emph{CMake}} --- кроссплатформенная система автоматизации сборки программного обеспечения
         из исходного кода}
    \end{itemize}
\end{frame}

%%%%%%%%%%%%%%%%%%%%%%%%%%%%%%%%%%%%%%%%%%%%%%%%%%%%%%%%%%%%%%%%%%%%%%%%%%%%%%%%

\begin{frame}[c]\frametitle{Алгоритм packJPG}

\vspace{\baselineskip}

\textbf{Основные этапы алгоритма}
\begin{itemize}
    \item{Декодирование JPEG}
    \item{Paeth предиктор}
    \item{Списки концов блоков}
    \item{Оптимизированное сканирование}
    \item{Арифметическое кодирование}
\end{itemize}

\end{frame}

\begin{frame}\frametitle{Анализ существующего кода}
    \textbf{Особенности кода проекта}
    \begin{itemize}
        \item{Более 10000 строк кода содержатся в 5 файлах}
        \item{Высокая цикломатическая сложность методов}
        \item{Процедурный стиль программирования}
    \end{itemize}
    \vspace{\baselineskip}
    \textbf{Основные этапы}
    \begin{itemize}
        \item{Выявление основных компонент программ}
        \item{Сопоставление описания алгоритма с его реализацией}
    \end{itemize}
\end{frame}


%%%%%%%%%%%%%%%%%%%%%%%%%%%%%%%%%%%%%%%%%%%%%%%%%%%%%%%%%%%%%%%%%%%%%%%%%%%%%%%%

\begin{frame}\frametitle{Рефакторинг}
    \begin{itemize}
        \item{ Реструктуризация исходного кода
            \begin{itemize}
                \item{Декомпозиция больших файлов}
                \item{Группировка компонент по области примения}
            \end{itemize}
        }
        \vspace{\baselineskip}
        \item{Основные недостатки
            \begin{itemize}
                \item{Дублирование кода}
                \item{Длинные методы}
                \item{Расходящиеся модификации}
            \end{itemize}
        }
    \end{itemize}
\end{frame}

%%%%%%%%%%%%%%%%%%%%%%%%%%%%%%%%%%%%%%%%%%%%%%%%%%%%%%%%%%%%%%%%%%%%%%%%%%%%%%%%

\begin{frame}\frametitle{Процедура параллельной модификации}
    \begin{itemize}
            \item{Поиск узких мест программы}
              \vspace{\baselineskip}
            \item{Инкрементальное программирование фрагментов кода
                \begin{itemize}
                    \item{Анализ возможности модификации}
                    \item{Параллельной реализации с помощью технологии OpenMP}
                    \item{Оценка проведнной модификации}
                \end{itemize}
            }
        \end{itemize}
\end{frame}

%%%%%%%%%%%%%%%%%%%%%%%%%%%%%%%%%%%%%%%%%%%%%%%%%%%%%%%%%%%%%%%%%%%%%%%%%%%%%%%%

 \begin{frame}[fragile]{Неудачные модификации}
        \begin{block}{Метод shift\_context}
            \begin{lstlisting}
void model_s::shift_context( int c )
{
    table_s* context;
    int i;
    ...
    for ( i = max_order; i > 0; i-- ) {
        context = contexts[ i - 1 ]->links[ c ];
        ...
        contexts[ i ] = context;
    }
}
            \end{lstlisting}
        \end{block}
\end{frame}

%%%%%%%%%%%%%%%%%%%%%%%%%%%%%%%%%%%%%%%%%%%%%%%%%%%%%%%%%%%%%%%%%%%%%%%%%%%%%%%%

 \begin{frame}[fragile]{Успешные модификации}
        \begin{block}{Параллельная реализация быстрой сортировки}
            \begin{lstlisting}
 template<class T>
     static void sort(T *a, int n) {
         long i = 0, j = n;
         float pivot = a[n / 2];
         ...
         #pragma omp task shared(a)
         if (j > 0) sort(a, j);
         #pragma omp task shared(a)
         if (n > i) sort(a + i, n - i);
         #pragma omp taskwait
     }

 }
            \end{lstlisting}
        \end{block}
\end{frame}
%%%%%%%%%%%%%%%%%%%%%%%%%%%%%%%%%%%%%%%%%%%%%%%%%%%%%%%%%%%%%%%%%%%%%%%%%%%%%%%%

\begin{frame}\frametitle{Оценка проведенных модификаций}
    Для оценки внесенных модификаций, были проведены тесты производительности.
    Сравнительная таблица результов работы до и после модификации представлена ниже:
    \begin{center}
        \begin{tabular}{ | p{4cm} | p{3cm} | p{3cm}|}
        \hline
        Размер изображения (пикс./мегабайт) & Исходная реализация $\t\pm\sigma$ (сек.) & Параллельная модификация $\t\pm\sigma$ (сек.) \\ \hline
        400x400/0.0136 & $0.027\pm0.008$ & $0.021\pm0.006$ \\ \hline
        1920x1080/1.075 & $1.387\pm0.02$ & $1.317\pm0.015$ \\ \hline
        4000x4000/3.500 & $5.22\pm0.09$ & $5.01\pm0.09$ \\ \hline
        12000x6000/14.500 & $19.93\pm0.25$ & $19.32\pm0.22$ \\ \hline
        \end{tabular}
    \end{center}
    Для оценки погрешности использовалось среднеквадратичное отклонение.
\end{frame}

%%%%%%%%%%%%%%%%%%%%%%%%%%%%%%%%%%%%%%%%%%%%%%%%%%%%%%%%%%%%%%%%%%%%%%%%%%%%%%%%

\begin{frame}\frametitle{Результаты}
\begin{itemize}
    \item{Основные части алгоритма не получилось распараллелить в связи с его спецификой}
    \item{После проведенных модификаций наблюдается прирост в среднем на 3-4 %}
}
\end{itemize}
\end{frame}

%%%%%%%%%%%%%%%%%%%%%%%%%%%%%%%%%%%%%%%%%%%%%%%%%%%%%%%%%%%%%%%%%%%%%%%%%%%%%%%%

\begin{frame}[c]
\begin{center}
{\Large \textcolor{blue}{Спасибо!}}
\end{center}
\end{frame}
    
%%%%%%%%%%%%%%%%%%%%%%%%%%%%%%%%%%%%%%%%%%%%%%%%%%%%%%%%%%%%%%%%%%%%%%%%%%%%%%%%
\end{document}